% ============================================================
%  Web Dev Cheatsheet — HTML · CSS · JS · Node.js · MongoDB
%  Compile with: pdflatex cheatsheet.tex
% ============================================================
\documentclass[10pt,a4paper,landscape]{article}

% ── Packages ────────────────────────────────────────────────
\usepackage[landscape, margin=1cm]{geometry}
\usepackage[utf8]{inputenc}
\usepackage[T1]{fontenc}
\usepackage[portuguese]{babel}
\usepackage{multicol}
\usepackage{xcolor}
\usepackage{listings}
\usepackage{booktabs}
\usepackage{tabularx}
\usepackage{array}
\usepackage{enumitem}
\usepackage{titlesec}
\usepackage{mdframed}
\usepackage{fontawesome5}
\usepackage{lmodern}
\usepackage{microtype}
\usepackage{parskip}
\usepackage{hyperref}

% ── Colours ─────────────────────────────────────────────────
\definecolor{bg}{HTML}{0F1117}
\definecolor{surface}{HTML}{1A1D27}
\definecolor{card}{HTML}{22263A}
\definecolor{accent}{HTML}{3B82F6}
\definecolor{htmlcol}{HTML}{E44D26}
\definecolor{csscol}{HTML}{6B8EF5}
\definecolor{jscol}{HTML}{F7DF1E}
\definecolor{nodecol}{HTML}{68A063}
\definecolor{mongocol}{HTML}{4DB33D}
\definecolor{shortcol}{HTML}{A78BFA}
\definecolor{codebg}{HTML}{0D1117}
\definecolor{codetext}{HTML}{79C0FF}
\definecolor{keyword}{HTML}{FF7B72}
\definecolor{string}{HTML}{A5D6FF}
\definecolor{comment}{HTML}{8B949E}
\definecolor{func}{HTML}{D2A8FF}
\definecolor{tipbg}{HTML}{1E3A2F}
\definecolor{tiptext}{HTML}{86EFAC}
\definecolor{warnbg}{HTML}{3A2A1E}
\definecolor{warntext}{HTML}{FCD34D}
\definecolor{muted}{HTML}{8892A4}
\definecolor{textcol}{HTML}{E2E8F0}

% ── Page style ──────────────────────────────────────────────
\pagecolor{bg}
\color{textcol}
\setlength{\columnsep}{8pt}
\setlength{\columnseprule}{0.4pt}
\def\columnseprulecolor{\color{card}}

% ── Code listings ───────────────────────────────────────────
\lstset{
  backgroundcolor=\color{codebg},
  basicstyle=\ttfamily\scriptsize\color{codetext},
  keywordstyle=\color{keyword}\bfseries,
  stringstyle=\color{string},
  commentstyle=\color{comment}\itshape,
  breaklines=true,
  breakatwhitespace=true,
  frame=single,
  framerule=0.4pt,
  rulecolor=\color{card},
  xleftmargin=4pt,
  xrightmargin=4pt,
  aboveskip=3pt,
  belowskip=3pt,
  showstringspaces=false,
  tabsize=2,
}
\lstdefinelanguage{JavaScript}{
  keywords={const,let,var,function,async,await,return,if,else,try,catch,new,class,import,export,from,of,for,while,true,false,null,undefined},
  morestring=[b]",
  morestring=[b]',
  morestring=[b]`,
  morecomment=[l]{//},
  morecomment=[s]{/*}{*/},
  sensitive=true
}

% ── Section titles ──────────────────────────────────────────
\titleformat{\section}[block]
  {\normalfont\bfseries\small}
  {}{0pt}
  {\colorbox{surface}{\parbox{\dimexpr\linewidth-2\fboxsep}{\strut #1\strut}}}
\titlespacing{\section}{0pt}{4pt}{2pt}

\titleformat{\subsection}[block]
  {\normalfont\bfseries\footnotesize\color{muted}}
  {}{0pt}{#1}
\titlespacing{\subsection}{0pt}{3pt}{1pt}

% ── Card environment ────────────────────────────────────────
\newmdenv[
  backgroundcolor=card,
  linecolor=surface,
  linewidth=0.6pt,
  innerleftmargin=5pt,
  innerrightmargin=5pt,
  innertopmargin=4pt,
  innerbottommargin=4pt,
  skipabove=3pt,
  skipbelow=3pt,
  roundcorner=3pt,
]{card}

% ── Tip / Warn boxes ────────────────────────────────────────
\newmdenv[
  backgroundcolor=tipbg,
  linecolor=mongocol,
  linewidth=1.5pt,
  topline=false, rightline=false, bottomline=false,
  innerleftmargin=5pt, innerrightmargin=4pt,
  innertopmargin=2pt, innerbottommargin=2pt,
  skipabove=2pt, skipbelow=2pt,
]{tipbox}

\newmdenv[
  backgroundcolor=warnbg,
  linecolor=warntext,
  linewidth=1.5pt,
  topline=false, rightline=false, bottomline=false,
  innerleftmargin=5pt, innerrightmargin=4pt,
  innertopmargin=2pt, innerbottommargin=2pt,
  skipabove=2pt, skipbelow=2pt,
]{warnbox}

% ── Table helpers ───────────────────────────────────────────
\newcolumntype{K}{>{\ttfamily\scriptsize\color{codetext}}l}
\newcolumntype{D}{>{\scriptsize}X}
\setlength{\tabcolsep}{4pt}
\renewcommand{\arraystretch}{1.15}

% ── Colour dots ─────────────────────────────────────────────
\newcommand{\cdot}[1]{\textcolor{#1}{\rule{6pt}{6pt}}\,}

% ── Helpers ─────────────────────────────────────────────────
\newcommand{\kbd}[1]{{\ttfamily\scriptsize\colorbox{surface}{\textcolor{codetext}{#1}}}}
\newcommand{\tip}[1]{\begin{tipbox}\scriptsize\textcolor{tiptext}{#1}\end{tipbox}}
\newcommand{\warn}[1]{\begin{warnbox}\scriptsize\textcolor{warntext}{#1}\end{warnbox}}

% ── Suppress paragraph indent ───────────────────────────────
\setlength{\parindent}{0pt}
\setlength{\parskip}{2pt}

% ============================================================
\begin{document}
\begin{multicols}{4}

% ── HEADER ──────────────────────────────────────────────────
{\centering
  {\large\bfseries\textcolor{accent}{⚡ Web Dev Cheatsheet}}\\[2pt]
  {\scriptsize\textcolor{muted}{HTML · CSS · JavaScript · Node.js · MongoDB}}\\[4pt]
}

% ════════════════════════════════════════════════════════════
%  HTML
% ════════════════════════════════════════════════════════════
\section{\textcolor{htmlcol}{■} HTML — Estrutura \& Semântica}

\begin{card}
\subsection{Estrutura Base (Emmet: \texttt{!} + Tab)}
\begin{lstlisting}[language=HTML]
<!DOCTYPE html>
<html lang="pt">
<head>
  <meta charset="UTF-8">
  <meta name="viewport"
    content="width=device-width,
             initial-scale=1.0">
  <title>Título</title>
</head>
<body>
</body>
</html>
\end{lstlisting}
\end{card}

\begin{card}
\subsection{Tags Semânticas HTML5}
\begin{tabularx}{\linewidth}{KD}
\toprule
Tag & Uso \\
\midrule
<header>  & Cabeçalho da página \\
<nav>     & Menu de navegação \\
<main>    & Conteúdo principal \\
<section> & Secção temática \\
<article> & Conteúdo independente \\
<aside>   & Barra lateral \\
<footer>  & Rodapé \\
<figure>  & Imagem com legenda \\
\bottomrule
\end{tabularx}
\end{card}

\begin{card}
\subsection{Emmet — Atalhos HTML}
\begin{tabularx}{\linewidth}{KD}
\toprule
Emmet & Resultado \\
\midrule
!              & Estrutura HTML completa \\
div.classe     & <div class="classe"> \\
ul>li*5        & ul com 5 li's \\
a[href=\#]     & <a href="\#"> \\
p\{Texto\}     & <p>Texto</p> \\
input:email    & input type="email" \\
nav>ul>li*4>a  & Nav com 4 links \\
h\$*3\{T \$\}  & h1, h2, h3 com texto \\
\bottomrule
\end{tabularx}
\end{card}

\begin{card}
\subsection{Formulário}
\begin{lstlisting}[language=HTML]
<form action="/enviar" method="POST">
  <label for="nome">Nome:</label>
  <input type="text" id="nome"
         name="nome" required>

  <label for="email">Email:</label>
  <input type="email" id="email"
         name="email" required>

  <button type="submit">Enviar</button>
</form>
\end{lstlisting}
\end{card}

\begin{card}
\subsection{Imagem Responsiva}
\begin{lstlisting}[language=HTML]
<picture>
  <source media="(min-width:800px)"
          srcset="grande.webp">
  <img src="pequena.jpg"
       alt="Descrição"
       loading="lazy"
       width="800" height="600">
</picture>
\end{lstlisting}
\tip{Usa sempre \texttt{loading="lazy"} para melhor performance.}
\end{card}

% ════════════════════════════════════════════════════════════
%  CSS
% ════════════════════════════════════════════════════════════
\section{\textcolor{csscol}{■} CSS — Estilo \& Layout}

\begin{card}
\subsection{Variáveis CSS}
\begin{lstlisting}[language=CSS]
:root {
  --cor-primaria: #3b82f6;
  --cor-texto:    #1e293b;
  --espacamento:  1rem;
  --radius:       8px;
}
.botao {
  background: var(--cor-primaria);
  padding: var(--espacamento);
  border-radius: var(--radius);
}
\end{lstlisting}
\end{card}

\begin{card}
\subsection{Flexbox}
\begin{lstlisting}[language=CSS]
.container {
  display: flex;
  justify-content: center; /* eixo X */
  align-items: center;     /* eixo Y */
  flex-wrap: wrap;
  gap: 1rem;
}
.item { flex: 1 1 200px; }
\end{lstlisting}
\tip{\texttt{gap} substitui margins entre itens flex.}
\end{card}

\begin{card}
\subsection{CSS Grid}
\begin{lstlisting}[language=CSS]
.grid {
  display: grid;
  grid-template-columns:
    repeat(auto-fill, minmax(250px, 1fr));
  gap: 1.5rem;
}
/* 3 colunas fixas */
.grid-3 {
  grid-template-columns: repeat(3, 1fr);
}
\end{lstlisting}
\end{card}

\begin{card}
\subsection{Media Queries (Mobile First)}
\begin{lstlisting}[language=CSS]
.card { width: 100%; }

@media (min-width: 768px) {
  .card { width: 48%; }
}
@media (min-width: 1024px) {
  .card { width: 32%; }
}
\end{lstlisting}
\begin{tabularx}{\linewidth}{KD}
480px  & Telemóvel \\
768px  & Tablet \\
1024px & Desktop \\
1280px & Ecrã grande \\
\end{tabularx}
\end{card}

\begin{card}
\subsection{Animações \& Transições}
\begin{lstlisting}[language=CSS]
.btn {
  transition: all 0.3s ease;
}
.btn:hover {
  transform: translateY(-3px);
  box-shadow: 0 10px 20px rgba(0,0,0,.2);
}
@keyframes fadeIn {
  from { opacity: 0; transform: translateY(20px); }
  to   { opacity: 1; transform: translateY(0); }
}
.el { animation: fadeIn 0.5s ease forwards; }
\end{lstlisting}
\end{card}

\begin{card}
\subsection{Seletores Avançados}
\begin{tabularx}{\linewidth}{KD}
\toprule
Seletor & Significado \\
\midrule
p:first-child    & Primeiro filho \\
li:nth-child(2n) & Elementos pares \\
a:not(.ativo)    & Excluir classe \\
input:focus      & Quando focado \\
div > p          & Filho direto \\
h2 + p           & Irmão adjacente \\
{[}data-x="y"{]} & Por atributo \\
::before         & Pseudo-elemento \\
\bottomrule
\end{tabularx}
\end{card}

\begin{card}
\subsection{Reset Moderno}
\begin{lstlisting}[language=CSS]
*, *::before, *::after {
  box-sizing: border-box;
  margin: 0;
  padding: 0;
}
\end{lstlisting}
\end{card}

% ════════════════════════════════════════════════════════════
%  JAVASCRIPT
% ════════════════════════════════════════════════════════════
\section{\textcolor{jscol}{■} JavaScript — Lógica \& DOM}

\begin{card}
\subsection{Selecionar Elementos}
\begin{lstlisting}[language=JavaScript]
const el  = document.querySelector('#id');
const els = document.querySelectorAll('.cls');

// Iterar
els.forEach(el => el.classList.add('ativo'));

// Manipular
el.textContent = 'Novo texto';
el.innerHTML   = '<strong>Bold</strong>';
el.style.color = 'red';
\end{lstlisting}
\end{card}

\begin{card}
\subsection{Eventos}
\begin{lstlisting}[language=JavaScript]
btn.addEventListener('click', (e) => {
  e.preventDefault();
  console.log('Clicado!');
});

// Delegação (eficiente)
document.addEventListener('click', (e) => {
  if (e.target.matches('.btn')) {
    // lógica aqui
  }
});
\end{lstlisting}
\end{card}

\begin{card}
\subsection{Fetch API}
\begin{lstlisting}[language=JavaScript]
// GET
async function getData() {
  try {
    const res  = await fetch('/api/dados');
    const data = await res.json();
    console.log(data);
  } catch (err) {
    console.error('Erro:', err);
  }
}

// POST
await fetch('/api/criar', {
  method: 'POST',
  headers: { 'Content-Type': 'application/json' },
  body: JSON.stringify({ nome: 'Ana' })
});
\end{lstlisting}
\end{card}

\begin{card}
\subsection{Desestruturação \& Spread}
\begin{lstlisting}[language=JavaScript]
// Arrays
const [a, b, ...resto] = [1, 2, 3, 4];

// Objetos
const { nome, idade = 18 } = utilizador;

// Spread
const novo = { ...obj1, ...obj2 };
const arr  = [...arr1, ...arr2];

// Default params
function saudar(nome = 'Mundo') {
  return `Olá, ${nome}!`;
}
\end{lstlisting}
\end{card}

\begin{card}
\subsection{Array — Métodos Essenciais}
\begin{lstlisting}[language=JavaScript]
const nums = [1, 2, 3, 4, 5];

nums.map(n => n * 2);       // [2,4,6,8,10]
nums.filter(n => n > 2);    // [3,4,5]
nums.reduce((a,n) => a+n, 0); // 15
nums.find(n => n > 3);      // 4
nums.some(n => n > 4);      // true
nums.every(n => n > 0);     // true
nums.includes(3);            // true
[...new Set(nums)];          // sem duplicados
\end{lstlisting}
\end{card}

\begin{card}
\subsection{LocalStorage}
\begin{lstlisting}[language=JavaScript]
// Guardar
localStorage.setItem('user',
  JSON.stringify({ nome: 'Ana' }));

// Ler
const user = JSON.parse(
  localStorage.getItem('user'));

// Remover
localStorage.removeItem('user');
localStorage.clear();
\end{lstlisting}
\warn{Nunca guardes senhas no localStorage!}
\end{card}

% ════════════════════════════════════════════════════════════
%  NODE.JS
% ════════════════════════════════════════════════════════════
\section{\textcolor{nodecol}{■} Node.js — Servidor \& Backend}

\begin{card}
\subsection{Servidor Express Básico}
\begin{lstlisting}[language=JavaScript]
const express = require('express');
const app = express();

app.use(express.json());

app.get('/', (req, res) => {
  res.json({ mensagem: 'Olá!' });
});

app.listen(3000, () => {
  console.log('http://localhost:3000');
});
\end{lstlisting}
\end{card}

\begin{card}
\subsection{Estrutura de Projeto}
\begin{lstlisting}
projeto/
├── src/
│   ├── routes/      # rotas da API
│   ├── models/      # modelos MongoDB
│   ├── controllers/ # lógica negócio
│   └── middleware/  # auth, etc.
├── .env
├── .gitignore
├── package.json
└── server.js
\end{lstlisting}
\tip{Instala: \texttt{npm init -y} → \texttt{npm i express mongoose dotenv}}
\end{card}

\begin{card}
\subsection{Variáveis de Ambiente}
\begin{lstlisting}
# .env
PORT=3000
MONGODB_URI=mongodb://localhost:27017/bd
JWT_SECRET=segredo_super_secreto
\end{lstlisting}
\begin{lstlisting}[language=JavaScript]
require('dotenv').config();
const port = process.env.PORT || 3000;
const uri  = process.env.MONGODB_URI;
\end{lstlisting}
\warn{Adiciona \texttt{.env} ao \texttt{.gitignore} sempre!}
\end{card}

\begin{card}
\subsection{CRUD com Express Router}
\begin{lstlisting}[language=JavaScript]
const router = require('express').Router();

router.get('/',    listarTodos);
router.get('/:id', obterUm);
router.post('/',   criar);
router.put('/:id', atualizar);
router.delete('/:id', eliminar);

module.exports = router;

// server.js
app.use('/api/utilizadores', router);
\end{lstlisting}
\end{card}

% ════════════════════════════════════════════════════════════
%  MONGODB
% ════════════════════════════════════════════════════════════
\section{\textcolor{mongocol}{■} MongoDB — Base de Dados NoSQL}

\begin{card}
\subsection{Ligação com Mongoose}
\begin{lstlisting}[language=JavaScript]
const mongoose = require('mongoose');

async function ligar() {
  try {
    await mongoose.connect(
      process.env.MONGODB_URI);
    console.log('MongoDB ligado!');
  } catch (err) {
    console.error(err);
    process.exit(1);
  }
}
ligar();
\end{lstlisting}
\end{card}

\begin{card}
\subsection{Schema \& Model}
\begin{lstlisting}[language=JavaScript]
const { Schema, model } = require('mongoose');

const schema = new Schema({
  nome:  { type: String, required: true },
  email: { type: String, required: true,
           unique: true, lowercase: true },
  idade: { type: Number, min: 0 },
  ativo: { type: Boolean, default: true }
}, { timestamps: true });

module.exports = model('User', schema);
\end{lstlisting}
\end{card}

\begin{card}
\subsection{CRUD — Operações}
\begin{lstlisting}[language=JavaScript]
const U = require('./models/User');

// CREATE
await U.create({ nome: 'Ana',
                 email: 'ana@mail.pt' });
// READ
await U.find({ ativo: true });
await U.findById(id);

// UPDATE
await U.findByIdAndUpdate(id,
  { nome: 'Ana Silva' }, { new: true });

// DELETE
await U.findByIdAndDelete(id);
\end{lstlisting}
\end{card}

\begin{card}
\subsection{Queries Avançadas}
\begin{lstlisting}[language=JavaScript]
// Filtros
await U.find({ idade: { $gte: 18 } });
await U.find({ nome: /ana/i });

// Ordenar, limitar, paginar
await U.find()
  .sort({ nome: 1 })   // 1=ASC -1=DESC
  .limit(10)
  .skip(20)
  .select('nome email');

// Contar
const total = await U.countDocuments();
\end{lstlisting}
\end{card}

\begin{card}
\subsection{Manual — Ligação Passo a Passo}
\begin{enumerate}[leftmargin=*, itemsep=1pt, topsep=1pt, label=\textcolor{mongocol}{\small\bfseries\arabic*.}]
  \item \textbf{Criar projeto:} \texttt{npm init -y}
  \item \textbf{Instalar:} \texttt{npm i express mongoose dotenv}
  \item \textbf{MongoDB Atlas:} criar cluster M0 gratuito em \texttt{mongodb.com/atlas}
  \item \textbf{.env:} colar a connection string
  \item \textbf{server.js:} \texttt{require('dotenv').config()} + \texttt{mongoose.connect(...)}
  \item \textbf{Testar:} \texttt{npm run dev} + Postman/Thunder Client
\end{enumerate}
\end{card}

% ════════════════════════════════════════════════════════════
%  ATALHOS DE TECLADO
% ════════════════════════════════════════════════════════════
\section{\textcolor{shortcol}{■} Atalhos de Teclado \& Formatações Rápidas}

\subsection{\textcolor{shortcol}{VS Code — Edição de Código}}
\begin{card}
\begin{tabularx}{\linewidth}{KD}
\toprule
Atalho & Ação \\
\midrule
Ctrl+/          & Comentar/descomentar linha \\
Alt+↑/↓         & Mover linha \\
Alt+Shift+↑/↓   & Duplicar linha \\
Ctrl+Shift+K    & Apagar linha inteira \\
Ctrl+Enter      & Nova linha abaixo \\
Ctrl+Shift+Enter & Nova linha acima \\
Ctrl+Z / Ctrl+Y & Desfazer / Refazer \\
Ctrl+X / Ctrl+C & Cortar/Copiar linha \\
\bottomrule
\end{tabularx}
\end{card}

\subsection{\textcolor{shortcol}{VS Code — Navegação \& Pesquisa}}
\begin{card}
\begin{tabularx}{\linewidth}{KD}
\toprule
Atalho & Ação \\
\midrule
Ctrl+P          & Abrir ficheiro rapidamente \\
Ctrl+Shift+P    & Paleta de comandos \\
Ctrl+F          & Pesquisar no ficheiro \\
Ctrl+H          & Substituir no ficheiro \\
Ctrl+Shift+F    & Pesquisar em todos os ficheiros \\
Ctrl+G          & Ir para linha número \\
Ctrl+Tab        & Alternar ficheiros abertos \\
Ctrl+\textbackslash & Dividir editor \\
\bottomrule
\end{tabularx}
\end{card}

\subsection{\textcolor{shortcol}{VS Code — Multi-cursor \& Seleção}}
\begin{card}
\begin{tabularx}{\linewidth}{KD}
\toprule
Atalho & Ação \\
\midrule
Ctrl+D          & Selecionar próxima ocorrência \\
Ctrl+Shift+L    & Selecionar TODAS as ocorrências \\
Alt+Click       & Adicionar cursor extra \\
Ctrl+Alt+↑/↓    & Cursor acima/abaixo \\
Shift+Alt+drag  & Seleção em coluna \\
Ctrl+L          & Selecionar linha inteira \\
Ctrl+Shift+[    & Fechar/Abrir bloco \\
\bottomrule
\end{tabularx}
\end{card}

\subsection{\textcolor{shortcol}{VS Code — Produtividade}}
\begin{card}
\begin{tabularx}{\linewidth}{KD}
\toprule
Atalho & Ação \\
\midrule
Ctrl+`          & Terminal integrado \\
Ctrl+B          & Mostrar/esconder sidebar \\
Alt+Shift+F     & Formatar documento \\
F2              & Renomear símbolo \\
F12             & Ir para definição \\
Alt+F12         & Pré-visualizar definição \\
Ctrl+Shift+V    & Pré-visualizar Markdown \\
Ctrl+K Z        & Modo Zen (ecrã cheio) \\
\bottomrule
\end{tabularx}
\end{card}

\subsection{\textcolor{shortcol}{Emmet — HTML Rápido}}
\begin{card}
\begin{tabularx}{\linewidth}{KD}
\toprule
Emmet & Resultado \\
\midrule
!                        & HTML completo \\
nav>ul>li*4>a            & Nav com 4 links \\
div.container>div.card*3 & 3 cards \\
table>tr*3>td*4          & Tabela 3×4 \\
ul>li.item\$*5           & Lista item1..item5 \\
section\#hero>h1+p+a.btn & Secção hero \\
\bottomrule
\end{tabularx}
\end{card}

\subsection{\textcolor{shortcol}{Emmet — CSS Rápido}}
\begin{card}
\begin{tabularx}{\linewidth}{KD}
\toprule
Abrev. & Propriedade \\
\midrule
df   & display: flex \\
dg   & display: grid \\
jcc  & justify-content: center \\
aic  & align-items: center \\
m:a  & margin: auto \\
p10  & padding: 10px \\
w100p & width: 100\% \\
bgc  & background-color \\
tac  & text-align: center \\
pos:r & position: relative \\
\bottomrule
\end{tabularx}
\end{card}

\subsection{\textcolor{shortcol}{Browser DevTools (F12)}}
\begin{card}
\begin{tabularx}{\linewidth}{KD}
\toprule
Atalho & Ação \\
\midrule
F12             & Abrir DevTools \\
Ctrl+Shift+I    & Abrir DevTools (alt.) \\
Ctrl+Shift+C    & Inspecionar elemento \\
Ctrl+Shift+J    & Consola JavaScript \\
Ctrl+Shift+M    & Modo responsivo \\
Ctrl+R          & Recarregar página \\
Ctrl+Shift+R    & Recarregar sem cache \\
Ctrl+U          & Ver código fonte \\
\bottomrule
\end{tabularx}
\end{card}

\subsection{\textcolor{shortcol}{Terminal / Linha de Comandos}}
\begin{card}
\begin{tabularx}{\linewidth}{KD}
\toprule
Comando & Ação \\
\midrule
Ctrl+C       & Parar processo \\
Ctrl+L       & Limpar terminal \\
↑ / ↓        & Histórico de comandos \\
Tab          & Auto-completar \\
Ctrl+A       & Início da linha \\
Ctrl+E       & Fim da linha \\
npm run dev  & Iniciar dev server \\
node -e "…"  & Executar JS direto \\
\bottomrule
\end{tabularx}
\end{card}

\vfill
{\centering\scriptsize\textcolor{muted}{Web Dev Cheatsheet · 2026 · Compilar com \texttt{pdflatex cheatsheet.tex}}\par}

\end{multicols}
\end{document}
